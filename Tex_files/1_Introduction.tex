\section{Введение}
Центры окраски в алмазах -- это атомы, внедрённые в кристаллическую решётку
алмаза. Наиболее интересными изученными центрами окраски являются $NV$-, $GeV$-
и $SiV$-центры. Уровни энергии электронов, связанных с этими центрами, имеют ряд
свойств, которые могут быть использованы в практических целях. Например, 
зависимость вида спектра люминесценции от температуры может быть основой для микроскопического
датчика температуры, который применяется в исследовании внутриклеточной активности
живых существ \cite{Therm}. Кроме того, такой датчик может быть использован
как замена существующих полупроводниковых термометров, в том числе в экстремальных
условиях. Алмаз с центрами окраски может быть использован для измерения магнитного
поля, что в перспективе открывает новые возможности для геологических исследований. 


