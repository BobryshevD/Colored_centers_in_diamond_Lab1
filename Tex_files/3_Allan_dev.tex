\section{Девиация Аллана}
Ясно, что в случае измерения температуры мы хотим получить максимально точный результат.
С точки зрения теории по виду спектра мы можем однозначно определить температуру, однако,
как и в любой другой физической системе, в алмазе имеются различные флуктуации и шумы 
температуры, механического напряжения и других параметров, влияющих на итоговый спектр.
Для учёта погрешности, вызванной влиянием шумов, используется девиация Аллана. 
Как известно, среднее значение шума данных равняется нулю, т.е. если провести 
бесконечное количество измерений и взять среднее, мы получим "чистый" результат.
При этом провести бесконечное число измерений невозможно, однако при увеличении
времени сбора данных влияние шумов уменьшается. 

Девиация Аллана имеет следующий смысл: фиксированный набор данных разбивается на равные
последовательные группы, значения в которых усредняются. Затем вычисляются разности квадратов 
этих средних между i-ой и i+1-ой группой для всех таких групп и усредняется по i,
формула \ref{Allan deviation eq}.
Получается вариация Аллана, которая зависит от разбиения на группы. Девиация Аллана 
представляет осбой квадратной корень из вариации Аллана. В логарифмическом масштабе
зависимость девиации Аллана от числа точек в каждой группе (времени усреднения данных)
является прямой. По наклону последней можно определить, какой именно тип шума присутствует
при измерениях.

\begin{equation}
    \label{Allan deviation eq}
    \sigma^2(\tau) = \frac{1}{2}\overline{\left(\overline{y_{i+1}}
    -\overline{y_{i}}\right)^2}
\end{equation}
